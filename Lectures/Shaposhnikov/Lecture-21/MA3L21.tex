\documentclass[12pt]{article}
\usepackage[left=1cm, right=1cm, top=2cm,bottom=1.5cm]{geometry} 

\usepackage[parfill]{parskip}
\usepackage[utf8]{inputenc}
\usepackage[T2A]{fontenc}
\usepackage[russian]{babel}
\usepackage{enumitem}
\usepackage[normalem]{ulem}
\usepackage{amsfonts, amsmath, amsthm, amssymb, mathtools}
\usepackage{tabularx}
\usepackage{hhline}

\usepackage{accents}
\usepackage{fancyhdr}
\pagestyle{fancy}
\renewcommand{\headrulewidth}{1.5pt}
\renewcommand{\footrulewidth}{1pt}

\usepackage{graphicx}
\usepackage[figurename=Рис.]{caption}
\usepackage{subcaption}
\usepackage{float}

%%Наименование папки откуда забирать изображения
\graphicspath{ {./images/} }

%%Изменение формата для ввода доказательства
\renewcommand{\proofname}{$\square$  \nopunct}
\renewcommand\qedsymbol{$\blacksquare$}

%%Изменение отступа на таблицах
\addto\captionsrussian{%
	\renewcommand{\proofname}{$\square$ \nopunct}%
}
%% Римские цифры
\newcommand{\RN}[1]{%
	\textup{\uppercase\expandafter{\romannumeral#1}}%
}

%% Для удобства записи
\newcommand{\MR}{\mathbb{R}}
\newcommand{\MC}{\mathbb{C}}
\newcommand{\MQ}{\mathbb{Q}}
\newcommand{\MN}{\mathbb{N}}
\newcommand{\MTB}{\mathbb{T}}
\newcommand{\MTI}{\mathbb{I}}
\newcommand{\MI}{\mathrm{I}}
\newcommand{\MJ}{\mathrm{J}}
\newcommand{\MH}{\mathrm{H}}
\newcommand{\MT}{\mathrm{T}}
\newcommand{\MU}{\mathcal{U}}
\newcommand{\MV}{\mathcal{V}}
\newcommand{\MB}{\mathcal{B}}
\newcommand{\MW}{\mathcal{W}}
\newcommand{\ML}{\mathcal{L}}
\newcommand{\VN}{\varnothing}
\newcommand{\VE}{\varepsilon}

\theoremstyle{definition}
\newtheorem{defn}{Опр:}
\newtheorem{rem}{Rm:}
\newtheorem{prop}{Утв.}
\newtheorem{exrc}{Упр.}
\newtheorem{lemma}{Лемма}
\newtheorem{theorem}{Теорема}
\newtheorem{corollary}{Следствие}

\newenvironment{cusdefn}[1]
{\renewcommand\thedefn{#1}\defn}
{\enddefn}

\DeclareRobustCommand{\divby}{%
	\mathrel{\text{\vbox{\baselineskip.65ex\lineskiplimit0pt\hbox{.}\hbox{.}\hbox{.}}}}%
}
%Короткий минус
\DeclareMathSymbol{\SMN}{\mathbin}{AMSa}{"39}
%Длинная шапка
\newcommand{\overbar}[1]{\mkern 1.5mu\overline{\mkern-1.5mu#1\mkern-1.5mu}\mkern 1.5mu}
%Функция знака
\DeclareMathOperator{\sgn}{sgn}

%Функция ранга
\DeclareMathOperator{\rk}{\text{rk}}

%Обозначение константы
\DeclareMathOperator{\const}{\text{const}}

\DeclareMathOperator*{\dsum}{\displaystyle\sum}
\newcommand{\ddsum}[2]{\displaystyle\sum\limits_{#1}^{#2}}

%Интеграл в большом формате
\DeclareMathOperator{\dint}{\displaystyle\int}
\newcommand{\ddint}[2]{\displaystyle\int\limits_{#1}^{#2}}
\newcommand{\ssum}[1]{\displaystyle \sum\limits_{n=1}^{\infty}{#1}_n}

\newcommand{\smallerrel}[1]{\mathrel{\mathpalette\smallerrelaux{#1}}}
\newcommand{\smallerrelaux}[2]{\raisebox{.1ex}{\scalebox{.75}{$#1#2$}}}

\newcommand{\smallin}{\smallerrel{\in}}
\newcommand{\smallnotin}{\smallerrel{\notin}}

\newcommand*{\medcap}{\mathbin{\scalebox{1.25}{\ensuremath{\cap}}}}%
\newcommand*{\medcup}{\mathbin{\scalebox{1.25}{\ensuremath{\cup}}}}%

\makeatletter
\newcommand{\vast}{\bBigg@{3.5}}
\newcommand{\Vast}{\bBigg@{5}}
\makeatother

%Промежуточное значение для sup\inf, поскольку они имеют разную высоту
\newcommand{\newsup}{\mathop{\smash{\mathrm{sup}}}}
\newcommand{\newinf}{\mathop{\mathrm{inf}\vphantom{\mathrm{sup}}}}

%Скалярное произведение
\DeclarePairedDelimiterX{\inner}[2]{\langle}{\rangle}{#1, #2}

%Подпись символов снизу
\newcommand{\ubar}[1]{\underaccent{\bar}{#1}}

%% Шапка для букв сверху
\newcommand{\wte}[1]{\widetilde{#1}}

%%Функция для обозначения равномерной сходимости по множеству
\newcommand{\uconv}[1]{\overset{#1}{\rightrightarrows}}
\newcommand{\uconvm}[2]{\overset{#1}{\underset{#2}{\rightrightarrows}}}


%%Функция для обозначения нижнего и верхнего интегралов
\def\upint{\mathchoice%
	{\mkern13mu\overline{\vphantom{\intop}\mkern7mu}\mkern-20mu}%
	{\mkern7mu\overline{\vphantom{\intop}\mkern7mu}\mkern-14mu}%
	{\mkern7mu\overline{\vphantom{\intop}\mkern7mu}\mkern-14mu}%
	{\mkern7mu\overline{\vphantom{\intop}\mkern7mu}\mkern-14mu}%
	\int}
\def\lowint{\mkern3mu\underline{\vphantom{\intop}\mkern7mu}\mkern-10mu\int}


\begin{document}
\lhead{Математический анализ - \RN{3}}
\chead{Шапошников С.В.}
\rhead{Лекция - 21}
\section*{Несобственные интегралы с параметром}

Пусть $X \neq \VN$, есть полуинтервал $[a,b)$, где возможно $b = +\infty$, есть функция $f \colon X \times [a,b) \to \MR (\MC)$. Теперь всегда будем подразумевать, что: $\forall x \in X, \, t \mapsto f(x,t)$ - интегрируема по Риману на $[a,c], \, \forall c \in [a,b)$. Рассмотрим следующую функцию:
$$
	F(x,y) = \ddint{a}{y}f(x,t)dt, \, F \colon X\times [a,b) \to \MR(\MC)
$$
\begin{defn}
	Если $\forall x \in X, \, \exists \, \lim\limits_{y\to b-}F(x,y)$, то этот предел обозначается:
	$$
		\ddint{a}{b}f(x,t)dt = \lim\limits_{y \to b-}F(x,y) = \lim\limits_{y \to b-}\ddint{a}{y}f(x,t)dt
	$$
	и называется \uwave{несобственным интегралом с параметром}.
\end{defn}
\begin{defn}
	Если $F(x,y) \uconvm{X}{y \to b-} \ddint{a}{b}f(x,t)dt$, то $\varphi(x) = \ddint{a}{b}f(x,t)dt$ и говорят что \uwave{несобственный интеграл} $\varphi(x)$ \uwave{сходится равномерно}.
\end{defn}
Равномерная сходимость есть далеко не всегда. Попробуем разобрать пример.

\textbf{Пример}: Сходится ли равномерно интеграл: $\ddint{0}{+\infty}e^{-(t-x)^2}dt, \, x \geq 0$? Рассмотрим разность:
$$
	\sup\limits_{x \geq 0}\left|\ddint{0}{+\infty}e^{-(t-x)^2}dt - \ddint{0}{y}e^{-(t-x)^2}dt\right| = \sup\limits_{x \geq 0} \ddint{y}{+\infty}e^{-(t-x)^2}dt \xrightarrow[y \to +\infty]{?}0
$$
$$
	\sup\limits_{x \geq 0} \ddint{y}{+\infty}e^{-(t-x)^2}dt \geq \ddint{y}{+\infty}e^{-(t-y)^2}dt = \ddint{0}{+\infty}e^{-s^2}ds = \dfrac{\sqrt{\pi}}{2} > 0
$$
Равномерной сходимости нет, при этом, если бы мы указали $0 \leq x \leq 10$, то она была бы:
$$
	y > 10 \Rightarrow \sup\limits_{x \in [0,10]} \ddint{y}{+\infty}e^{-(t-x)^2}dt \leq \ddint{y}{+\infty}e^{-(t-10)^2}dt
$$
\begin{rem}
	Заметим, что как и для рядов за сходимость здесь отвечает ``хвост'' интеграла. 	Пусть несобственный интеграл $\ddint{a}{b}f(x,t)dt$ сходится $\forall x \in X$. Тогда его равномерная сходимость равносильна следующему:
	$$
		\sup\limits_{x \in X}\left|\ddint{y}{b}f(x,t)dt \right| \xrightarrow[y \to b-]{} 0
	$$
	Иногда это называют \uwave{супремум-критерием}. Рассмотрим эту сходимость подробнее:
	$$
		\varphi(x) - F(x,y) = \ddint{a}{b}f(x,t)dt - F(x,y) = \ddint{a}{b}f(x,t)dt - \ddint{a}{y}f(x,t)dt = \ddint{y}{b}f(x,t)dt \uconvm{X}{y \to b-} 0
	$$
	где последнее верно в силу следующего:
	$$
		F(x,y) \uconvm{X}{y \to b-}\ddint{a}{b}f(x,t)dt \Leftrightarrow \sup\limits_{x \in X}\left|F(x,y) - \ddint{a}{b}f(x,t)dt \right| =  \sup\limits_{x \in X}\left|F(x,y) - \varphi(x) \right| \xrightarrow[y\to b-]{} 0
	$$
\end{rem}

\begin{theorem}(\textbf{критерий Коши})
	Несобственный интеграл $\ddint{a}{b}f(x,t)dt$ - сходится равномерно на $X \Leftrightarrow$
	выполнено условие Коши:
	$$
		\forall \VE > 0, \, \exists \, c \in (a,b) \colon \forall c_1, c_2 \in (c,b), \, \sup\limits_{x \in X} \left|\ddint{c_1}{c_2}f(x,t)dt\right| < \VE
	$$
\end{theorem}
\begin{proof}
	Заметив что:
	$$
		\ddint{c_1}{c_2}f(x,t)dt = F(x,c_2) - F(x, c_1)
	$$
	теорема следует сразу из критерия Коши равномерной сходимости для семейства функций.
\end{proof}
\begin{rem}
	Критерий Коши чаще всего используется для доказательства того, что что-то не сходится. Для положительных результатов, в типичных ситуациях, есть признаки сходимости. Обычно подбирают участки, сколь угодно близко к $b$, где несобственный интеграл не является равномерно малым. Есть одно наблюдение, которое следует из критерия Коши, которое во многих случаях позволяет в одну строчку показать, что какой-то интеграл не сходится равномерно.
\end{rem}
\begin{corollary}
	Пусть $X \subset Y$ - метрическое пространство, $x_0$ - предельная точка пространства $X$, $x_0\notin X$ и $\forall c \in [a,b),\, f(x_0,t)$ - интегрируема на $[a,c]$, где $|f(x,t)| \leq C$ и верно $f(x,t) \xrightarrow[x \to x_0]{} f(x_0,t)$. Тогда, если интеграл $\ddint{a}{b}f(x_0,t)dt$ не сходится, то $\ddint{a}{b}f(x,t)dt$ не сходится равномерно на $X$.
\end{corollary}
\begin{proof}
	Пусть $\ddint{a}{b}f(x,t)dt$ сходится равномерно на $X$, тогда по критерию Коши:
	$$
		\forall \VE > 0, \, \exists \, c \in (a,b) \colon \forall c_1, c_2 \in (c,b), \, \sup\limits_{x \in X} \left|\ddint{c_1}{c_2}f(x,t)dt\right| < \VE \Rightarrow \forall x \in X, \, \left|\ddint{c_1}{c_2}f(x,t)dt\right| < \VE  
	$$
	Мы находимся в условиях теоремы Арцела $\Rightarrow$ можем переходить к пределу:
	$$
		\left|\ddint{c_1}{c_2}f(x,t)dt\right| < \VE  \Rightarrow \lim\limits_{x \to x_0}\left|\ddint{c_1}{c_2}f(x,t)dt\right| = \left|\ddint{c_1}{c_2}f(x_0,t)dt\right| \leq \VE  
	$$
	Значит интеграл $\ddint{a}{b}f(x_0,t)dt$ - сходится $\Rightarrow$ получили противоречие.
\end{proof}

\textbf{Пример}: Сходится ли равномерно интеграл: $\ddint{0}{1}\dfrac{dt}{t^x}$ при $x \in (0,1)$? Очевидно, что при каждом $x \in (0,1)$ этот интеграл сходится, поскольку степень меньше $1$, но равномерной сходимости не будет:
$$
	x = 1 \Rightarrow \ddint{0}{1}\dfrac{dt}{t} = \ln{t}\Big|_{t = 0}^{1} = \ln{1} - \lim\limits_{t \to 0+}\ln{t} = +\infty
$$
Данный метод называется \uwave{методом граничной точки}.

\begin{theorem}(\textbf{признак Вейерштрасса})
	\begin{enumerate}[label=\arabic*)]
		\item Если $|f(x,t)| \leq g(x,t)$ и $\ddint{a}{b}g(x,t)dt$ - сходится равномерно на $X$, то $\ddint{a}{b}f(x,t)dt$ тоже сходится равномерно на $X$;
		\item Если $|f(x,t)| \leq g(t)$ и $\ddint{a}{b}g(t)dt$ - сходится, то $\ddint{a}{b}f(x,t)dt$ сходится равномерно на $X$;
	\end{enumerate}
\end{theorem}
\begin{proof}
	$2)$ является частным случаем $1)$. Докажем $1)$: по критерию Коши:
	$$
		\forall \VE > 0, \, \exists \, c \in (a,b) \colon \forall c_1, c_2 \in (c,b), \, \sup\limits_{x \in X} \left|\ddint{c_1}{c_2}g(x,t)dt\right| < \VE
	$$
	$$
		\left| \ddint{c_1}{c_2}f(x,t)dt \right| \leq \ddint{c_1}{c_2}g(x,t)dt \Rightarrow \sup\limits_{x \in X}\left| \ddint{c_1}{c_2}f(x,t)dt \right| \leq \sup\limits_{x \in X}\ddint{c_1}{c_2}g(x,t)dt < \VE
	$$
\end{proof}

\textbf{Пример}: Будет ли равномерно сходится интеграл: $\ddint{1}{+\infty}\dfrac{\sin{(x^2 + tx + t^2)}}{x^2 + t^2} dt$ по $x$ ? Будет, поскольку:
$$
	\forall x \in \MR, \, \left|\dfrac{\sin{(x^2 + tx + t^2)}}{x^2 + t^2} \right| \leq \dfrac{1}{t^2},\, \ddint{1}{+\infty}\dfrac{dt}{t^2} = -\dfrac{1}{t}\Big|_{t = 1}^{+\infty} = 1 < \infty
$$

\newpage
\subsection*{Равномерная сходимость интеграла от произведения}
Рассмотрим интегралы следующего вида: $\ddint{a}{b}f(x,t)g(x,t)dt$. Когда он будет сходиться?
Аналогичные вопросы (и ответы) уже возникали при изучении сходимости рядов и несобственных интегралов.
\begin{prop}
	Если $\ddint{a}{b}|f(x,t)|dt$ - сходится равномерно на $X$, $g(x,t)$ - равномерно ограничена на $X$, то есть: 
	$$
		\exists \, C > 0 \colon \forall x, \, \forall t, \, |g(x,t)| \leq C
	$$
	Тогда $\ddint{a}{b}f(x,t)g(x,t)dt$ сходится равномерно на $X$.
\end{prop}
\begin{proof}
	Поскольку $|f(x,t)g(x,t)| \leq C |f(x,t)|$, то по признаку Вейерштрасса сразу получаем требуемое.
\end{proof}

\begin{prop}
	Пусть интеграл $\ddint{a}{y}|f(x,t)|dt$ - равномерно ограничен на $X$, то есть:
	$$
		\exists \, C > 0 \colon \forall x, \, \forall y, \, \ddint{a}{y}|f(x,t)|dt \leq C
	$$
	и функция $g(x,t) \uconvm{X}{t \to b-} 0$. Тогда $\ddint{a}{b}f(x,t)g(x,t)dt$ сходится равномерно на $X$.
\end{prop}
\begin{proof}
	По критерию Коши, пусть $\VE > 0$ и мы нашли $c \in (a,b)$ такое, что:
	$$
		\forall t \in (c,b), \, \sup\limits_{x \in X}|g(x,t)| < \VE
	$$
	Будем отсюда рассматривать $c_1, c_2 \colon c < c_1 < c_2 < b$.
	Оценим разность:
	$$
		\left|\ddint{c_1}{c_2}f(x,t)g(x,t)dt\right| \leq \VE \ddint{c_1}{c_2}|f(x,t)|dt \leq \VE \ddint{a}{c_2}|f(x,t)|dt \leq \VE {\cdot}C
	$$
	Условие Коши выполнено $\Rightarrow$ произведение сходится равномерно на $X$.
\end{proof}

\subsection*{Интегрирование по частям}

\begin{lemma}
	Пусть $\forall x \in X$, функция $t \mapsto f(x,t)$ - непрерывна на $[a,b)$, функция $t \mapsto g(x,t)$ - непрерывно дифференцируема на $[a,b)$. Введем обозначение: $F(x,t) = \ddint{a}{t}f(x,s)ds - h(x)$, где $h(x)$ - произвольная функция от $x$ (не зависит от $t$). Тогда:
	$$
		\ddint{a}{y}f(x,t)g(x,t)dt = g(x,y)F(x,y) - g(x,a)F(x,a) - \ddint{a}{y}g_t^\prime(x,t)F(x,t)dt =
	$$
	$$
		= g(x,y){\cdot}\left(\ddint{a}{y}f(x,s)ds - h(x)\right) + g(x,a)h(x) - \ddint{a}{y}g_t^\prime(x,t)\left(\ddint{a}{t}f(x,s)ds - h(x)\right)
	$$
\end{lemma}
\begin{proof}
	Вспоминая производную интеграла по верхнему пределу, мы получим:
	$$
		\ddint{a}{y}f(x,t)g(x,t)dt = \ddint{a}{y}g(x,t){\cdot}\left(F(x,t)\right)_t^\prime dt = g(x,y)F(x,y) - g(x,a)F(x,a) - \ddint{a}{y}g_t^\prime(x,t)F(x,t)dt
	$$
	где последнее равенство это формула интегрирования по частям для несобственных интегралов. Подставляя выражение для $F(x,t)$ получаем требуемое.
\end{proof}

\begin{theorem}
	Если $\exists\, h(x) \colon g(x,y)F(x,y)$ сходится равномерно на $X$ при $y \to b-$, то интегралы: 
	$$	
		\ddint{a}{b}f(x,t)g(x,t)dt, \; \ddint{a}{b}g_t^\prime(x,t)F(x,t)dt
	$$
	или оба сходятся равномерно на $X$, или оба не сходятся равномерно на $X$.
\end{theorem}
\begin{proof}
	Следует из арифметики пределов и формулы интегрирования по частям: $g(x,a)F(x,a)$ - константа по $y$, $g(x,y)F(x,y)$ - сходится равномерно $\Rightarrow$ если сходится левая часть в формуле интегрирования по частям, то будет сходится правая и наоборот.
\end{proof}

\begin{corollary}(\textbf{признаки Абеля-Дирихле}) Пусть функция $t \mapsto f(x,t)$ - непрерывна на $[a,b]$,\\ функция $t \mapsto g(x,t)$ - непрерывно дифференцируема на $[a,b)$ и верно: $g_t^\prime(x,t) \leq 0 \, (\geq 0), \, \forall t \in [a,b)$.
	\begin{enumerate}[label=\arabic*)]
		\item \uwave{\textbf{Признак Дирихле}}: $\ddint{a}{y}f(x,t)dt$ - равномерно ограниченны на $X$ и $g(x,t) \uconvm{X}{t \to b-} 0$. Тогда интеграл от произведения $\ddint{a}{b}f(x,t)g(x,t)dt$  сходится равномерно на $X$;
		\item \uwave{\textbf{Признак Абеля}}: $\ddint{a}{b}f(x,t)dt$ - сходится равномерно на $X$, $g(x,t)$ - равномерно ограниченна на $X$. Тогда интеграл от произведения $\ddint{a}{b}f(x,t)g(x,t)dt$  сходится равномерно на $X$;
	\end{enumerate}
\end{corollary}

\begin{rem}
	Заметим схожесть формулировок с утверждениями $1$ и $2$ с тем отличием, что здесь за счет монотонности не требуются модули (абсолютной сходимости) под интегралом.
\end{rem}
\begin{proof}\hfill
	\begin{enumerate}[label=\arabic*)]
		\item Пусть $h \equiv 0 \Rightarrow F(x,y) = \ddint{a}{y}f(x,s)ds$, тогда: $g(x,y) F(x,y) = g(x,t)\ddint{a}{y}f(x,s)ds \uconvm{X}{y \to b-} 0$	по арифметике пределов (см. лекцию $10$), поскольку $F(x,y)$ - равномерно ограничена, а $g(x,y) \uconvm{X}{y \to b-} 0$. По теореме $3$ достаточно исследовать равномерную сходимость интеграла: 
		$$
			\ddint{a}{b}g_t^\prime(x,t)F(x,t)dt = \ddint{a}{b}g_t^\prime(x,t)\left(\ddint{a}{t}f(x,s)ds\right)dt
		$$
		где $F(x,t)$ - равномерно ограничена, а  $\ddint{a}{b}\left|g_t^\prime(x,t)\right|dt$ будет сходиться равномерно (из-за монотонности) $\Rightarrow$ используем формулу Ньютона-Лейбница, тогда получим:
		$$
			\ddint{a}{y}\left|g_t^\prime(x,t)\right|dt = -\ddint{a}{y}g_t^\prime(x,t)dt = g(x,a) - g(x,y) \uconvm{X}{y\to b-} g(x,a)
		$$ 
		Следовательно, применим утверждение $1$ и тем самым докажем требуемое;
		
		\item Пусть мы определим $h(x)$ так:
		$$
			h(x) = \ddint{a}{b}f(x,t)dt \Rightarrow F(x,y) = \ddint{a}{y}f(x,s)ds - \ddint{a}{b}f(x,s)ds
		$$
		тогда: $F(x,y) \uconvm{X}{y \to b-} 0$, поскольку интеграл $h(x)$ сходится равномерно по условию. Так как по условию $g(x,t)$ равномерно ограничена на $X$, то мы получим: $g(x,y)F(x,y) \uconvm{X}{y\to b-} 0$. По теореме $3$ необходимо исследовать интеграл: $\ddint{a}{b}g_t^\prime(x,t) F(x,t)dt$ на сходимость. Рассмотрим:
		$$
			\ddint{a}{y}\left|g_t^\prime(x,t)\right|dt = -\ddint{a}{y}g_t^\prime(x,t)dt = g(x,a) - g(x,y)
		$$
		где опять воспользовались формулой Ньютона-Лейбница. Зная, что $g(x,y)$ равномерно ограниченна на $X$, то $-g(x,y) + g(x,a)$ также будет равномерно ограниченной на $X$ $\Rightarrow$ равномерно ограничен и интеграл от модуля производной. Применяем утверждение $2$ и получим требуемое.
	\end{enumerate}
\end{proof}

\begin{rem}
	В признаках Абеля-Дирихле функция $F(x,t)$ может принимать комплексные значения, а функция $g(x,t)$ - нет.
\end{rem}
\newpage
\subsection*{Примеры использования признаков сходимости}

\textbf{Пример}: Сходится ли равномерно интеграл: $\ddint{0}{+\infty}\dfrac{\sin{t}}{t+x^2}dt, \, x \geq 1$ ? Будет ли поточечная сходимость? 
\begin{proof}
	Первообразная синуса ограничена, $\tfrac{1}{t + x^2}$ монотонно стремится к нулю $\Rightarrow$ работает признак Дирихле для поточечной сходимости. Попробуем применить его к равномерной сходимости:
	$$
		\forall x \in X, \, \ddint{0}{y}\sin{t}\, dt = \cos{0} - \cos{y} \leq 2; \, \dfrac{1}{t + x^2} \uconvm{X}{t \to \infty} 0
	$$
	Следовательно, по признаку Дирихле есть равномерная сходимость. Если взять модуль: $|\sin{t}|$, то даже обычной сходимости не будет.
\end{proof}

\textbf{Пример}: Сходится ли равномерно интеграл: $\ddint{0}{+\infty}\dfrac{\sin{t}}{t}e^{-xt}dt, \, x > 0$ ?
\begin{proof}
	Сходится равномерно по признаку Абеля: без экспоненты получаем константу по $x$ (т.е. сходимость равномерная), а $e^{-xt}$ - равномерно ограниченна и монотонна.
\end{proof}

\textbf{Пример}: Сходится ли равномерно интеграл: $\ddint{0}{+\infty}\dfrac{\sin{(xt)}}{t}dt, \, x > 0$ ?
\begin{proof}
	Видим, что у функции две особенности, поэтому подход должен быть следующим - разбить интеграл на два и отдельно исследовать в $0$, и отдельно в $+\infty$:
	$$
		\ddint{0}{+\infty}\dfrac{\sin{(xt)}}{t}dt = \ddint{0}{1}\dfrac{\sin{(xt)}}{t}dt + \ddint{1}{+\infty}\dfrac{\sin{(xt)}}{t}dt
	$$
	Вообще-то небезопасно делать замены в интеграле, зависящие от параметра. Для примера, сделаем преобразование здесь:
	$$
		\ddint{0}{+\infty}\dfrac{\sin{(xt)}}{t}dt = \ddint{0}{+\infty}\dfrac{\sin{(xt)}}{xt}d(xt) = \ddint{0}{+\infty}\dfrac{\sin{u}}{u}du
	$$
	Теперь интеграл не зависит от параметра, но отсюда не следует, что интеграл сходится равномерно. Это интеграл Дирихле. Ранее, мы уже находили его (лекция $8$) Попробуем рассмотреть изначальное разбиение. Очевидно, что второй интеграл сходится по признаку Дирихле для обычной сходимости и, более того, это значение не равно $0$. Если он ещё сходится и равномерно на полуинтервале $[1, +\infty)$, то хвост интеграла должен сходиться равномерно к $0$, проверим это:
	$$
		\ddint{c}{+\infty}\dfrac{\sin{(xt)}}{t}dt = \ddint{xc}{+\infty}\dfrac{\sin{u}}{u}du \Rightarrow \forall c, \, \exists \, x = \dfrac{b}{c} > 0 \Rightarrow \ddint{xc}{+\infty}\dfrac{\sin{u}}{u}du = \ddint{b}{+\infty}\dfrac{\sin{u}}{u}du 
	$$
	Поскольку мы знаем, что интеграл Дирихле положителен для $x > 0$, то подберём $b$ так, чтобы:
	$$
		\left|\ddint{b}{+\infty}\dfrac{\sin{u}}{u}du\right| = |C| > 0 \Rightarrow \sup\limits_{x > 0}\left|\ddint{c}{+\infty}\dfrac{\sin{(xt)}}{t}dt\right| \geq \left|\ddint{b}{+\infty}\dfrac{\sin{u}}{u}du\right| = |C| \nrightarrow 0
	$$
	Таким образом, мы получаем, что второй интеграл не сходится равномерно на $X = (0,+\infty)$. Аналогично, можно установить такой же факт для первого интеграла. 
\end{proof}

Рассмотрим ещё пример, показывающий что небезопасно делать замены, зависящие от параметра.

\textbf{Пример}: Сходится ли равномерно интеграл: $\ddint{0}{+\infty}xe^{-xt}dt, \, x > 0$ ?
\begin{proof}
	Делаем замену:
	$$
		\ddint{0}{+\infty}xe^{-xt}dt = \ddint{0}{+\infty}e^{-xt}d(xt) = \ddint{0}{+\infty}e^{-u}du
	$$
	Здесь особенность только одна, в $+\infty$. Попробуем изучить равномерную сходимость честно:
	$$
		\ddint{c}{+\infty}xe^{-xt}dt = \ddint{xc}{+\infty}e^{-u}du
	$$
	В хвосте мы можем делать замену. Попробуем испортить сходимость хвоста, выбирая параметр $x$:
	$$
		x = \dfrac{1}{c} \Rightarrow \ddint{1}{+\infty}e^{-u}du = \dfrac{1}{e} > 0
	$$
	Это не приближается к $0$, следовательно равномерной сходимости нет.
\end{proof}

\textbf{Пример}: Сходится ли равномерно интеграл: $\ddint{0}{+\infty}e^{-xt}\sin{t} \, dt, \, x > 0$ ? 
\begin{proof}
	Видно, что $e^{-xt} \not\uconv{}0$, например: 
	$$
		x = \dfrac{1}{t} \Rightarrow e^{-xt} = \dfrac{1}{e}
	$$
	Следовательно признак Дирихле использовать нельзя. Попробуем метод граничной точки. Пусть есть равномерная сходимость. Функции ограничены, всё интегрируемо, можно применять теорему Арцела:
	$$
		\ddint{0}{y}e^{-xt}\sin{t} \, dt \xrightarrow[x \to 0]{} \ddint{0}{y}\sin{t} \, dt \Rightarrow \ddint{0}{+\infty}\sin{t} \, dt \text{ - сходится}
	$$
	Но последнее не может быть верным $\Rightarrow$ получили противоречие $\Rightarrow$ нет равномерной сходимости. Можно то же самое показать с помощью критерия Коши: выбрать промежуток, где синус больше $\tfrac{1}{2}$, загнать $x$ близко к $0$, чтобы этот множитель оказался не меньше, чем $\tfrac{1}{2} \Rightarrow$ получим интеграл, не меньший, чем $\tfrac{1}{4}$ умноженный на фиксированную длину промежутка.
\end{proof}
\begin{rem}
	В общей ситуации, есть интеграл $\ddint{a}{b}f(x,t)dt$ - сходится равномерно на $X' = X \setminus \{x_0\}$, пусть мы умеем переходить к пределу:
	$$
		\ddint{a}{y}f(x,t)dt \xrightarrow[x \to x_0]{} \ddint{a}{y}f(x_0,t)dt \Rightarrow \ddint{a}{b}f(x_0,t)dt \text{ - сходится}
	$$
	Это следует из критерия Коши, поскольку оно будет верно для всех $x \in X'$, то устремим $x$ к $x_0$ и получим то же самое неравенство, при $x = x_0$: 
	$$
		\sup\limits_{x \in X'}\left|\ddint{c_1}{c_2}f(x,t)dt\right| < \VE \Rightarrow \lim\limits_{x \to x_0}\left|\ddint{c_1}{c_2}f(x,t)dt\right|=  \left|\ddint{c_1}{c_2}f(x_0,t)dt\right| \leq \VE \Rightarrow \ddint{a}{b}f(x_0,t)dt \text{ - сходится}
	$$
\end{rem}

\begin{exrc}
	Построить пример, когда $\ddint{a}{y}f(x,t)dt$ - равномерно ограничен, $\forall x, \, \exists \, t(x) \colon g_t^\prime \leq 0$ на $[t(x), + \infty)$, но при этом $\ddint{a}{b}f(x,t)g(x,t)dt$ не сходится равномерно. (См. задачник ВОС).
\end{exrc}
\begin{proof}
	Рассмотрим следующий интеграл:
	$$
 		\ddint{0}{+\infty}\dfrac{\cos{(t - x)}}{(t-x)^2 + 1}dt, \, x \in [0,+\infty) \Rightarrow g(x,t) = \dfrac{1}{(t-x)^2 + 1}, \, f(x,t) = \cos{(t-x)} \Rightarrow
	$$
	$$
		\Rightarrow \ddint{0}{y}f(x,t)dt = \ddint{0}{y}\cos{(t-x)}dt = \sin{(y - x)} + \sin{x} \leq 2, \, \forall x,y \in [0, +\infty)
	$$
	$$
		\Rightarrow \forall x,t \in [0, +\infty), \, g(x,t) \leq 1, \, g_t^\prime(x,t) = -\dfrac{2(t-x)}{\left((t-x)^2 + 1\right)^2} \Rightarrow \forall x \geq 0, \, \forall t \geq x, \, g_t^\prime \leq 0
	$$
	То есть функция $g(x,t)$ при каждом $x \geq 0$ убывает на $[t(x),+\infty) = [x, + \infty)$. Покажем, что интеграл произведения не сходится равномерно на $[0,+\infty)$ через критерий Коши. Пусть $c > 0$, тогда выберем:
	$$
		c_1 = c + 1, \, c_2 = c + 3, \, x = c + 2 \Rightarrow c_2 > c_1 > c > 0 \Rightarrow
	$$
	$$
		\left|\ddint{c_1}{c_2}\dfrac{\cos{(t- c - 2)}}{(t - c - 2)^2 + 1}dt\right| = \left|\ddint{-1}{1}\dfrac{\cos{u}}{u^2 +1 }du\right| \geq \dfrac{1}{2}{\cdot}\left|\ddint{-1}{1}\cos{u} \, du\right| = \sin{1} = \VE > 0
	$$
	Таким образом, получаем что критерий Коши нарушается и равномерной сходимости нет.
\end{proof}

\end{document}