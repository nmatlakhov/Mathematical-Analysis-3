\documentclass[12pt]{article}
\usepackage[left=1cm, right=1cm, top=2cm,bottom=1.5cm]{geometry} 

\usepackage[parfill]{parskip}
\usepackage[utf8]{inputenc}
\usepackage[T2A]{fontenc}
\usepackage[russian]{babel}
\usepackage{enumitem}
\usepackage[normalem]{ulem}
\usepackage{amsfonts, amsmath, amsthm, amssymb, mathtools}
\usepackage{tabularx}
\usepackage{hhline}

\usepackage{accents}
\usepackage{fancyhdr}
\pagestyle{fancy}
\renewcommand{\headrulewidth}{1.5pt}
\renewcommand{\footrulewidth}{1pt}

\usepackage{graphicx}
\usepackage[figurename=Рис.]{caption}
\usepackage{subcaption}
\usepackage{float}

%%Наименование папки откуда забирать изображения
\graphicspath{ {./images/} }

%%Изменение формата для ввода доказательства
\renewcommand{\proofname}{$\square$  \nopunct}
\renewcommand\qedsymbol{$\blacksquare$}

%%Изменение отступа на таблицах
\addto\captionsrussian{%
	\renewcommand{\proofname}{$\square$ \nopunct}%
}
%% Римские цифры
\newcommand{\RN}[1]{%
	\textup{\uppercase\expandafter{\romannumeral#1}}%
}

%% Для удобства записи
\newcommand{\MR}{\mathbb{R}}
\newcommand{\MC}{\mathbb{C}}
\newcommand{\MQ}{\mathbb{Q}}
\newcommand{\MN}{\mathbb{N}}
\newcommand{\MTB}{\mathbb{T}}
\newcommand{\MTI}{\mathbb{I}}
\newcommand{\MI}{\mathrm{I}}
\newcommand{\MJ}{\mathrm{J}}
\newcommand{\MH}{\mathrm{H}}
\newcommand{\MT}{\mathrm{T}}
\newcommand{\MU}{\mathcal{U}}
\newcommand{\MV}{\mathcal{V}}
\newcommand{\MB}{\mathcal{B}}
\newcommand{\MW}{\mathcal{W}}
\newcommand{\ML}{\mathcal{L}}
\newcommand{\VN}{\varnothing}
\newcommand{\VE}{\varepsilon}

\theoremstyle{definition}
\newtheorem{defn}{Опр:}
\newtheorem{rem}{Rm:}
\newtheorem{prop}{Утв.}
\newtheorem{exrc}{Упр.}
\newtheorem{lemma}{Лемма}
\newtheorem{theorem}{Теорема}
\newtheorem{corollary}{Следствие}

\newenvironment{cusdefn}[1]
{\renewcommand\thedefn{#1}\defn}
{\enddefn}

\DeclareRobustCommand{\divby}{%
	\mathrel{\text{\vbox{\baselineskip.65ex\lineskiplimit0pt\hbox{.}\hbox{.}\hbox{.}}}}%
}
%Короткий минус
\DeclareMathSymbol{\SMN}{\mathbin}{AMSa}{"39}
%Длинная шапка
\newcommand{\overbar}[1]{\mkern 1.5mu\overline{\mkern-1.5mu#1\mkern-1.5mu}\mkern 1.5mu}
%Функция знака
\DeclareMathOperator{\sgn}{sgn}

%Функция ранга
\DeclareMathOperator{\rk}{\text{rk}}

%Обозначение константы
\DeclareMathOperator{\const}{\text{const}}

\DeclareMathOperator*{\dsum}{\displaystyle\sum}
\newcommand{\ddsum}[2]{\displaystyle\sum\limits_{#1}^{#2}}

%Интеграл в большом формате
\DeclareMathOperator{\dint}{\displaystyle\int}
\newcommand{\ddint}[2]{\displaystyle\int\limits_{#1}^{#2}}
\newcommand{\ssum}[1]{\displaystyle \sum\limits_{n=1}^{\infty}{#1}_n}

\newcommand{\smallerrel}[1]{\mathrel{\mathpalette\smallerrelaux{#1}}}
\newcommand{\smallerrelaux}[2]{\raisebox{.1ex}{\scalebox{.75}{$#1#2$}}}

\newcommand{\smallin}{\smallerrel{\in}}
\newcommand{\smallnotin}{\smallerrel{\notin}}

\newcommand*{\medcap}{\mathbin{\scalebox{1.25}{\ensuremath{\cap}}}}%
\newcommand*{\medcup}{\mathbin{\scalebox{1.25}{\ensuremath{\cup}}}}%

\makeatletter
\newcommand{\vast}{\bBigg@{3.5}}
\newcommand{\Vast}{\bBigg@{5}}
\makeatother

%Промежуточное значение для sup\inf, поскольку они имеют разную высоту
\newcommand{\newsup}{\mathop{\smash{\mathrm{sup}}}}
\newcommand{\newinf}{\mathop{\mathrm{inf}\vphantom{\mathrm{sup}}}}

%Скалярное произведение
\DeclarePairedDelimiterX{\inner}[2]{\langle}{\rangle}{#1, #2}

%Подпись символов снизу
\newcommand{\ubar}[1]{\underaccent{\bar}{#1}}

%% Шапка для букв сверху
\newcommand{\wte}[1]{\widetilde{#1}}

%%Функция для обозначения равномерной сходимости по множеству
\newcommand{\uconv}[1]{\overset{#1}{\rightrightarrows}}

%%Функция для обозначения нижнего и верхнего интегралов
\def\upint{\mathchoice%
	{\mkern13mu\overline{\vphantom{\intop}\mkern7mu}\mkern-20mu}%
	{\mkern7mu\overline{\vphantom{\intop}\mkern7mu}\mkern-14mu}%
	{\mkern7mu\overline{\vphantom{\intop}\mkern7mu}\mkern-14mu}%
	{\mkern7mu\overline{\vphantom{\intop}\mkern7mu}\mkern-14mu}%
	\int}
\def\lowint{\mkern3mu\underline{\vphantom{\intop}\mkern7mu}\mkern-10mu\int}


\begin{document}
\lhead{Математический анализ - \RN{3}}
\chead{Шапошников С.В.}
\rhead{Лекция - 14}
\section*{Функциональные ряды}
\begin{defn}
	Ряд $\displaystyle \sum\limits_{n = 1}^{\infty}f_n(x)$ \uwave{сходится поточечно на} $X \Leftrightarrow \forall x \in X, \, \displaystyle \sum\limits_{n = 1}^{\infty}f_n(x)$ - сходится.
\end{defn}
\begin{defn}
	Ряд $\displaystyle \sum\limits_{n = 1}^{\infty}f_n(x)$ \uwave{сходится равномерно на} $X \Leftrightarrow $ последовательность $S_N(x) = \displaystyle\sum\limits_{n = 1}^N f_n(x)$ частичных сумм этого ряда  сходится равномерно на $X$.
\end{defn}
\begin{prop}(\textbf{необходимое условие равномерной сходимости ряда})
	Если ряд $\displaystyle \sum\limits_{n = 1}^{\infty}f_n(x)$ сходится равномерно на $X$, то его слагаемые равномерно стремятся к нулю: $f_n \uconv{X}0$.
\end{prop}

\begin{theorem}(\textbf{критерий Коши равномерной сходимости ряда})
	Ряд $\displaystyle \sum\limits_{n = 1}^{\infty}f_n(x)$ сходится равномерно на $X$ тогда и только тогда, когда:
	$$
		\forall \VE > 0, \, \exists \, N \colon \forall n,m > N, \, \sup\limits_{X}\left| \sum\limits_{k = m}^{n} f_k(x)\right| < \VE
	$$
\end{theorem}
\begin{theorem}(\textbf{признак Вейерштрасса})
	Пусть $f_n \colon X \to \MR$ и $\exists \, a_n \geq 0 \colon \forall x \in X, \, |f_n(x)| \leq a_n$, тогда из сходимости ряда $\displaystyle \sum\limits_{n = 1}^{\infty}a_n$ следует равномерная сходимость ряда $\displaystyle \sum\limits_{n = 1}^{\infty}f_n(x)$.
\end{theorem}

\begin{theorem}(\textbf{обобщение признака Вейерштрасса})
	Пусть $\ddsum{n = 1}{\infty}f_n(x)$, и $\ddsum{n=1}{\infty}g_n(x)$ - два функциональных ряда, где последний сходится равномерно на $X$. Пусть также справедливо следующее: 
	$$
		\forall x \in X,\, |f_n(x)| \leq g_n(x)
	$$ 
	тогда ряд $\ddsum{n = 1}{\infty}f_n(x)$ будет сходится равномерно.
\end{theorem}
\begin{proof}
	Проверим критерий Коши:
	$$
		\forall x \in X, \, \left| \sum\limits_{k = m}^{n} f_k(x)\right| 	\leq \sum\limits_{k = m}^n g_k(x) \Rightarrow \sup\limits_{X}\left| \sum\limits_{k = m}^{n} f_k(x)\right| \leq \sup\limits_{X}\left(\sum\limits_{k = m}^n g_k(x)\right)
	$$
	Поскольку для ряда $\displaystyle \sum\limits_{n = 1}^{\infty}g_k(x)$ критерий Коши выполняется и оценка сверху не зависит от $x$, то критерий Коши для рассматриваемого функционального ряда сразу выполняется.
\end{proof}

\newpage
\subsection*{Признак Дини}
\begin{theorem}(\textbf{признак Дини})
	Пусть $K$ - компакт в метрическом пространстве, $f_n, f \in C(K)$, числовая последовательность $|f_n(x) - f(x)|$ - невозрастает по $n$ и стремится к нулю $\forall x \in K$. Тогда $f_n \uconv{K} f$.
\end{theorem}
\begin{rem}
	Таким образом, если мы взяли последовательность непрерывных функций, знаем, что они поточечно сходятся к непрерывной функции, то из этого не следует равномерная сходимость. Что нужно добавить для равномерной сходимости? Надо добавить монотонность (и компактность). Более того, достаточно добавить монотонность либо по $n$, либо по $x \Rightarrow$ любая монотонность, компактность и непрерывность будут вести к равномерной сходимости.	
\end{rem}
\begin{rem}
	Данный признак неудобен тем, что необходимо точно знать что $f$ -  непрерывная функция, поэтому на практике его применяют не так часто. Но иногда применяется и этот признак.
\end{rem}

\textbf{Типичный пример}: Пусть есть ряд $\ddsum{ n = 1 }{\infty} g_n(x)  = S(x)$, знаем, что $g_n(x), S(x) \in C(K), \, g_n \geq 0$. Ряд из $g_n$ сходится, тогда:
$$
	\ddsum{ n = 1 }{N} g_n(x) \uconv{K} S(x)
$$
Также с помощью этого признака можно доказывать отсутствие непрерывности у предельной функции, если знаем, что нет равномерной сходимости, но при этом мы рассматриваем функции на компакте. Как показать, что $S$ не может быть непрерывной - от противного, пусть $S$ - непрерывная, тогда по признаку Дини ряд сойдется равномерно, но это не так $\Rightarrow$ функция не является непрерывной функцией.

\begin{rem}
	Отметим, что наличие равномерной сходимости это достаточное условие, но не необходимое.  
\end{rem}
\begin{proof}
	Введем функции $h_n(x)$:
	$$
		h_n(x) = |f_n(x) - f(x)| \geq 0, \,  \forall x \in K, \, h_n(x) \xrightarrow[n \to \infty]{} 0, \, h_n(x) \geq h_{n+1}(x)
	$$
	надо доказать, что $h_n(x) \uconv{} 0$. Пусть $x \in K$, возьмем $\VE> 0$, поскольку $h_n(x) \xrightarrow[n \to \infty]{} 0$, то:
	$$
		\exists \, N_x \colon h_{N_x}(x) < \VE
	$$
	Так как, $h_{N_x}$ - непрерывна в точке $x$, то найдется окрестность $\MU(x)$, такая что $\forall y \in \MU(x), \, h_{N_x}(y) < 2\VE$:
	$$
		\forall x \in K, \, h_{N_x}(x) \in C(K) \Rightarrow \exists \, \MU(x) \subset K \colon \forall y \in \MU(x), \, |h_{N_x}(x) - h_{N_x}(y)| < \VE \Rightarrow h_{N_x}(y) < 2\VE
	$$
	Из-за монотонности, мы получим:
	$$
		\forall n \geq N_x, \, \forall y \in \MU(x), \, h_n(y) \leq h_{N_x}(y) < 2\VE 
	$$
	Поскольку $K$ - компакт, то объединение таких окрестностей по каждому $x \in K$ покроет $K$:
	$$
		K \subset \bigcup\limits_{x \in K} \MU(x) \Rightarrow \exists \, \MU(x_1), \dotsc, \MU(x_M) \colon K \subset \bigcup\limits_{k = 1}^M \MU(x_k)
	$$
	где последнее верно в силу компактности $K$: $\exists$ конечный набор окрестностей покрывающих $K$. На каждой из этих окрестностей есть свой номер $N_{x_1}, \dotsc, N_{x_M}$, начиная с которого все соответствующие функции $h_n(x)$ становятся меньше $2\VE$. Возьмем $N = \max\{N_{x_1}, \dotsc, N_{x_M}\}$, тогда:
	$$
		\forall n > N, \, \forall y \in K, \, h_n(y) < 2\VE
	$$
	поскольку $\exists \, j \colon y \in \MU(x_j) \wedge N \geq N_{x_j}$. Таким образом, требуемое доказано.
\end{proof}
\begin{rem}
	От компактности в этой теореме отказаться невозможно, пример $x^n$.
\end{rem}

\newpage
\subsection*{Равномерная сходимость рядов вида $\ddsum{n = 1}{\infty}a_n(x)b_n(x)$
}
Будем исследовать равномерную сходимость в рядах следующего вида:
$$
	\ddsum{n = 1}{\infty}a_n(x)b_n(x)
$$
Сразу заметим, что если мы знаем про равномерную сходимость одного из рядов $a_n$ или $b_n$, то интересно понять, какие нужны свойства для второго ряда, чтобы сходилось произведение $a_n b_n$.

\begin{prop}
	Пусть $\ddsum{n = 1}{\infty}|a_n(x)|$ сходится равномерно на $X$. Если $b_n(x)$ - равномерно ограничены на $X$:
	$$
		\exists \, C \colon \forall x \in X, \, \forall n, \, |b_n(x)| \leq C
	$$
	тогда ряд $\ddsum{n = 1}{\infty}a_n(x)b_n(x)$ сходится равномерно.
\end{prop}
\begin{proof}
	Следует из обобщенного признакак Вейерштрасса, поскольку:
	$$
		\forall x \in X, \, |a_n(x) b_n(x)| \leq C|a_n(x)|
	$$
	Следовательно исходный ряд сходится равномерно.
\end{proof}

\begin{prop}
	Пусть частичные суммы $\ddsum{n = 1}{N}|a_n(x)|$ - равномерно ограничены на $X$ и $b_n \uconv{X} 0$, тогда ряд произведения $\ddsum{n = 1}{\infty}a_n(x)b_n(x)$ сходится равномерно.
\end{prop}
\begin{proof}
	Проверяем критерий Коши:
	$$
		\left|\sum\limits_{k = m+1}^{n}a_k(x)b_k(x)\right| \leq \sup\limits_{x \in X \wedge m + 1 \leq k < n} |b_k(x)|{\cdot} \!\!\sum\limits_{k = m+1}^{n}|a_k(x)| \leq \sup\limits_{x \in X \wedge m + 1 \leq k < n} |b_k(x)|{\cdot} C 
	$$
	Поскольку $b_k \uconv{X} 0$, то начиная с некоторого номера все точные верхние грани $b_k(x)$ будут меньше наперед заданного $\VE$, то есть:
	$$
		\forall \VE > 0, \, \exists \, N \colon \forall m > N, \, \sup\limits_{x \in X}|b_m(x)| < \VE 
	$$
	Пусть $m,n > N$, тогда:
	$$
		\left|\sum\limits_{k = m+1}^{n}a_k(x)b_k(x)\right|\leq \sup\limits_{x \in X \wedge m + 1 \leq k < n} |b_k(x)|{\cdot} C < \VE C
	$$
\end{proof}
Это достаточно простые наблюдения, которые мы сможем усилить далее с помощью преобразования Абеля.
\newpage
\subsection*{Преобразование Абеля}

Рассмотрим следующую сумму $\ddsum{n = 1}{N}a_nb_n$ (не важно, числа это или функции). Аналогично случаю для простых рядов, введем новые переменные: $B_n = b_1 + \dotsc + b_n, \, B_0 = 0$, тогда:
$$
	\ddsum{n = 1}{N}a_n b_n = \ddsum{n = 1}{N}a_n (B_n - B_{n-1}) = (a_1 B_1 - a_1 B_0) + (a_2 B_2 - a_2 B_1) + \dotsc + (a_N B_n - a_N B_{N-1}) = 
$$
$$
	= (a_1 - a_2)B_1 + \dotsc + (a_{N-1} - a_N) B_{N-1} + a_N B_N - a_1B_0 = \ddsum{n = 1}{N - 1}(a_n - a_{n+1})B_n - a_1 B_0 + a_N B_N
$$
Мы это уже делали ранее. Только ради технического удобства, проделаем следующее:
$$
	\ddsum{n = 1}{N}a_n (B_n - B_{n-1}) = \ddsum{n = 1}{N}a_n ((B_n -B)  - (B_{n-1}-B)) = \ddsum{n = 1}{N - 1}(a_n - a_{n+1})(B_n - B) - a_1 (B_0 - B) + a_N (B_N - B)
$$
Тогда будет верно следующее равенство:
$$
	\ddsum{n = 1}{N}a_n(x) b_n(x) = \ddsum{n = 1}{N - 1}(a_n(x) - a_{n+1}(x)){\cdot}(B_n(x) - B(x)) + a_1(x){\cdot} B(x) + a_N(x){\cdot} (B_N(x) - B(x))
$$

Исходя из этого равенства, можно сформулировать следующую теорему.
\begin{theorem}
	Пусть функции $a_n(x), b_n(x)$ определены на $X$ и существует такая функция $B(x)$ на $X$, что функции $a_N(x)(B_N(x) - B(x))$ равномерно сходятся на $X$ (обозначения взяты из преобразования Абеля). Тогда ряды: 
	$$
		\ddsum{n = 1}{\infty}a_n(x) b_n(x) \text{ и } \ddsum{n = 1}{\infty}(a_n(x) - a_{n+1}(x))\left(B_n(x) - B(x)\right)
	$$ 
	одновременно сходятся или расходятся равномерно на $X$.
\end{theorem}
\begin{proof}
	Очевидно следует из преобразования Абеля.
\end{proof}

\begin{corollary}(\textbf{признаки Абеля-Дирихле}):
	\begin{itemize}
		\item[1)] \textbf{Признак Дирихле}: Если $a_n(x) \uconv{X} 0$, $\forall x \in X$ последовательность $a_n(x)$ - монотонна и последовательность $B_n(x) =b_1 (x)  + \dotsc + b_n(x)$ - равномерно ограничена на $X$, то ряд $\ddsum{n = 1}{\infty}a_n(x)b_n(x)$ сходится на $X$ равномерно;
		
		\item [2)] \textbf{Признак Абеля}: Если $a_n(x)$ равномерно ограничены на $X$, $\forall x\in X$ числовая последовательность $\{a_n(x)\}$ монотонна и ряд $\ddsum{n = 1}{\infty}b_n(x)$ сходится равномерно на $X$, то ряд $\ddsum{n = 1}{\infty}a_n(x)b_n(x)$ сходится на $X$ равномерно;
	\end{itemize}
\end{corollary}
\begin{rem}
	То есть, берем обычные признаки Абеля-Дирихле и везде где можно вставляем ``равномерно'', получаются признаки Абеля-Дирихле равномерной сходимости.
\end{rem}
\begin{proof}\hfill
	\begin{enumerate}[label={\arabic*)}]
		\item \textbf{Признак Дирихле}: Применим теорему с функцией $B \equiv 0$. Поскольку $B_n(x)$ - равномерно ограничены, а последовательность $a_n \uconv{X} 0$, то по утверждению из лекции $10$: 
		$$
			a_N(x)B_N(x) \uconv{X} 0
		$$ 
		Тогда:
		$$
	 		\ddsum{n = 1}{\infty}a_n(x)b_n(x) \text{ - сходится равномерно} \Leftrightarrow \ddsum{n = 1}{\infty} \left(a_n(x) - a_{n + 1}(x)\right){\cdot} B_n(x) \text{ - сходится равномерно}
		$$
		Рассмотрим следующий ряд: $\ddsum{n = 1}{\infty} |a_n(x) - a_{n + 1}(x)|$ и его частичные суммы. Поскольку последовательность $a_n$ - монотонна, то мы знаем, как раскроется модуль при фиксированном $x$, тогда:
		$$
			\ddsum{n = 1}{N} \left|a_n(x) - a_{n+1}(x)\right| = \pm\left(a_1(x) - a_2(x) + a_2(x) - a_3(x) + \dotsc + a_N(x) - a_{N+1}(x)\right) = 
		$$
		$$	
			= \pm(a_1(x) - a_{N+1}(x)) = \left|a_1(x) - a_{N+1}(x)\right| \uconv{X} |a_1(x)|
		$$
		где последнее верно в силу того, что:
		$$
			\left| |a_1(x)| - \left|a_1(x) - a_{N+1}(x)\right| \right| \leq \left|a_1(x) - a_1(x) + a_{N+1}(x)\right| \leq \sup\limits_{x \in X} \left|a_{N+1}(x)\right| \to 0
		$$
		то есть модуль разности сходится к $|a_1(x)|$ равномерно по определению равномерной сходимости. Тогда:
		$$
			\ddsum{n = 1}{N}|a_n(x) - a_{n+1}(x)| \uconv{X} |a_1(x)|
		$$
		то есть равномерно сходящаяся последовательность из модулей умножается на равномерно ограниченную $\Rightarrow$ по утверждению $2$ получаем равномерно сходящийся ряд $\Rightarrow$ сходится исходный ряд. 
		\item \textbf{Признак Абеля}: Применим теорему с функцией $B(x) = \ddsum{n = 1}{\infty}b_n(x)$. Поскольку $a_n(x)$ - равномерно ограничены, а сходимость ряда $b_n$ это тоже самое, что и сходимость его частичных сумм (см. предыдущую лекцию про сходимость хвостов), то: 
		$$
			B_N(x) - B(x) \uconv{X} 0 \Rightarrow a_N{\cdot}(B_N(x) - B(x)) \uconv{X} 0
		$$
		Тогда:
		$$
			\ddsum{n = 1}{\infty}a_n(x)b_n(x) \text{ - сх. равномерно} \Leftrightarrow \ddsum{n = 1}{\infty} \left(a_n(x) - a_{n + 1}(x)\right){\cdot} \left(B_n(x) - B(x)\right) \text{ - сх. равномерно}
		$$
		Используя монтонность и равномерную ограниченность $a_n$, мы получим:
		$$
			\forall N, \, \forall x, \, \ddsum{n = 1}{N}|a_n(x) - a_{n+1}(x)| = \left|a_1(x) - a_{N+1}(x)\right| \leq C
		$$
		то есть частичные суммы равномерно ограничены. Поскольку $|B_N(x) - B(x)| \uconv{X} 0$ (см. предыдущую лекцию про сходимость хвостов), то по утвержденю $3$ равномерно сходится ряд: 
		$$
			\ddsum{n = 1}{\infty} \left(a_n(x) - a_{n + 1}(x)\right){\cdot} \left(B_n(x) - B(x)\right) 
		$$ 
		Следовательно, сходится исходный ряд.
	\end{enumerate}	
\end{proof}

\textbf{Пример}: Рассмотрим следующий типичный пример для применения признаков:
$$
	\ddsum{n = 1}{\infty}\dfrac{\sin{nx}}{n}
$$
Рассмотрим два случая:
\begin{enumerate}[label={\arabic*)}]
	\item $0 < \delta < x < 2\pi - \delta$, в этом случае понятно, что $a_n(x) = \tfrac{1}{n}$ - монотонная и равномерно стремится к нулю. Функции $b_n(x) = \sin{nx}$. Необходимо понять, что суммы $\ddsum{n = 1}{N}\sin{nx}$ - равномерны ограничены:
	$$
		2\sin{\tfrac{x}{2}}\sin{kx} = \cos{\left(k - \tfrac{1}{2}\right)x} - \cos{\left(k +  \tfrac{1}{2}\right)x} 
	$$
	Тогда эти косинусы будут сокращаться в следующей сумме:
	$$
		2\sin{\tfrac{x}{2}}\ddsum{n = 1}{N}\sin{nx} = \cos{\tfrac{x}{2}} - \cos{\left(N+\tfrac{1}{2}\right)x} \Rightarrow \ddsum{n = 1}{N}\sin{nx} = \dfrac{\cos{\tfrac{x}{2}} - \cos{\left(N+\tfrac{1}{2}\right)x}}{2\sin{\tfrac{x}{2}}}
	$$
	Теперь необходимо найти равномерную оценку для этой суммы (одновременно и для $N$, и для $x$). Из условия:
	$$
		\dfrac{\delta}{2} < \dfrac{x}{2} < \pi - \dfrac{\delta}{2} \Rightarrow \sin{\tfrac{x}{2}} \geq \sin{\tfrac{\delta}{2}} \Rightarrow \left|\ddsum{n = 1}{N}\sin{nx}\right| \leq \dfrac{2}{2\sin{\tfrac{\delta}{2}}} = \dfrac{1}{\sin{\tfrac{\delta}{2}}}
	$$
	По признаку Дирихле ряд сходится равномерно;
	\item $0 < x <2\pi$, в этом случае воспользуемся критерием Коши и распишем следующую сумму:
	$$
		\ddsum{n = m+1}{2m}\dfrac{\sin{nx}}{n}
	$$
	Выберем $x = \tfrac{1}{2m}$, тогда:
	$$
		\forall n = \overline{m + 1, 2m}, \, \dfrac{1}{2} \leq nx \leq 1 \Rightarrow  \sin{nx} > \sin{\dfrac{1}{2}} \Rightarrow \ddsum{n = m+1}{2m}\dfrac{\sin{nx}}{n} \geq \dfrac{\sin{\tfrac{1}{2}}}{2}
	$$
	Поскольку сумма по $a_n$ будет тоже больше $\tfrac{1}{2}$. Таким образом, какое бы далекое $m$ не взяли, заданная сумма будет выше фиксированного значения, а это опровергает условие Коши $\Rightarrow$ ряд не сходится равномерно;
		
\end{enumerate}


\end{document}