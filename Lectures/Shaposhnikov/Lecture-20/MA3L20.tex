\documentclass[12pt]{article}
\usepackage[left=1cm, right=1cm, top=2cm,bottom=1.5cm]{geometry} 

\usepackage[parfill]{parskip}
\usepackage[utf8]{inputenc}
\usepackage[T2A]{fontenc}
\usepackage[russian]{babel}
\usepackage{enumitem}
\usepackage[normalem]{ulem}
\usepackage{amsfonts, amsmath, amsthm, amssymb, mathtools}
\usepackage{tabularx}
\usepackage{hhline}

\usepackage{accents}
\usepackage{fancyhdr}
\pagestyle{fancy}
\renewcommand{\headrulewidth}{1.5pt}
\renewcommand{\footrulewidth}{1pt}

\usepackage{graphicx}
\usepackage[figurename=Рис.]{caption}
\usepackage{subcaption}
\usepackage{float}

%%Наименование папки откуда забирать изображения
\graphicspath{ {./images/} }

%%Изменение формата для ввода доказательства
\renewcommand{\proofname}{$\square$  \nopunct}
\renewcommand\qedsymbol{$\blacksquare$}

%%Изменение отступа на таблицах
\addto\captionsrussian{%
	\renewcommand{\proofname}{$\square$ \nopunct}%
}
%% Римские цифры
\newcommand{\RN}[1]{%
	\textup{\uppercase\expandafter{\romannumeral#1}}%
}

%% Для удобства записи
\newcommand{\MR}{\mathbb{R}}
\newcommand{\MC}{\mathbb{C}}
\newcommand{\MQ}{\mathbb{Q}}
\newcommand{\MN}{\mathbb{N}}
\newcommand{\MTB}{\mathbb{T}}
\newcommand{\MTI}{\mathbb{I}}
\newcommand{\MI}{\mathrm{I}}
\newcommand{\MJ}{\mathrm{J}}
\newcommand{\MH}{\mathrm{H}}
\newcommand{\MT}{\mathrm{T}}
\newcommand{\MU}{\mathcal{U}}
\newcommand{\MV}{\mathcal{V}}
\newcommand{\MB}{\mathcal{B}}
\newcommand{\MW}{\mathcal{W}}
\newcommand{\ML}{\mathcal{L}}
\newcommand{\VN}{\varnothing}
\newcommand{\VE}{\varepsilon}

\theoremstyle{definition}
\newtheorem{defn}{Опр:}
\newtheorem{rem}{Rm:}
\newtheorem{prop}{Утв.}
\newtheorem{exrc}{Упр.}
\newtheorem{lemma}{Лемма}
\newtheorem{theorem}{Теорема}
\newtheorem{corollary}{Следствие}

\newenvironment{cusdefn}[1]
{\renewcommand\thedefn{#1}\defn}
{\enddefn}

\DeclareRobustCommand{\divby}{%
	\mathrel{\text{\vbox{\baselineskip.65ex\lineskiplimit0pt\hbox{.}\hbox{.}\hbox{.}}}}%
}
%Короткий минус
\DeclareMathSymbol{\SMN}{\mathbin}{AMSa}{"39}
%Длинная шапка
\newcommand{\overbar}[1]{\mkern 1.5mu\overline{\mkern-1.5mu#1\mkern-1.5mu}\mkern 1.5mu}
%Функция знака
\DeclareMathOperator{\sgn}{sgn}

%Функция ранга
\DeclareMathOperator{\rk}{\text{rk}}

%Обозначение константы
\DeclareMathOperator{\const}{\text{const}}

\DeclareMathOperator*{\dsum}{\displaystyle\sum}
\newcommand{\ddsum}[2]{\displaystyle\sum\limits_{#1}^{#2}}

%Интеграл в большом формате
\DeclareMathOperator{\dint}{\displaystyle\int}
\newcommand{\ddint}[2]{\displaystyle\int\limits_{#1}^{#2}}
\newcommand{\ssum}[1]{\displaystyle \sum\limits_{n=1}^{\infty}{#1}_n}

\newcommand{\smallerrel}[1]{\mathrel{\mathpalette\smallerrelaux{#1}}}
\newcommand{\smallerrelaux}[2]{\raisebox{.1ex}{\scalebox{.75}{$#1#2$}}}

\newcommand{\smallin}{\smallerrel{\in}}
\newcommand{\smallnotin}{\smallerrel{\notin}}

\newcommand*{\medcap}{\mathbin{\scalebox{1.25}{\ensuremath{\cap}}}}%
\newcommand*{\medcup}{\mathbin{\scalebox{1.25}{\ensuremath{\cup}}}}%

\makeatletter
\newcommand{\vast}{\bBigg@{3.5}}
\newcommand{\Vast}{\bBigg@{5}}
\makeatother

%Промежуточное значение для sup\inf, поскольку они имеют разную высоту
\newcommand{\newsup}{\mathop{\smash{\mathrm{sup}}}}
\newcommand{\newinf}{\mathop{\mathrm{inf}\vphantom{\mathrm{sup}}}}

%Скалярное произведение
\DeclarePairedDelimiterX{\inner}[2]{\langle}{\rangle}{#1, #2}

%Подпись символов снизу
\newcommand{\ubar}[1]{\underaccent{\bar}{#1}}

%% Шапка для букв сверху
\newcommand{\wte}[1]{\widetilde{#1}}

%%Функция для обозначения равномерной сходимости по множеству
\newcommand{\uconv}[1]{\overset{#1}{\rightrightarrows}}
\newcommand{\uconvm}[2]{\overset{#1}{\underset{#2}{\rightrightarrows}}}


%%Функция для обозначения нижнего и верхнего интегралов
\def\upint{\mathchoice%
	{\mkern13mu\overline{\vphantom{\intop}\mkern7mu}\mkern-20mu}%
	{\mkern7mu\overline{\vphantom{\intop}\mkern7mu}\mkern-14mu}%
	{\mkern7mu\overline{\vphantom{\intop}\mkern7mu}\mkern-14mu}%
	{\mkern7mu\overline{\vphantom{\intop}\mkern7mu}\mkern-14mu}%
	\int}
\def\lowint{\mkern3mu\underline{\vphantom{\intop}\mkern7mu}\mkern-10mu\int}


\begin{document}
\lhead{Математический анализ - \RN{3}}
\chead{Шапошников С.В.}
\rhead{Лекция - 20}
\section*{Равномерная сходимость семейства функций}
Пусть $X \neq \VN$ (непустое множество), $Y$ - метрическое пространство, $y_0$ - предельная точка $Y$ и обозначим через $Y^\prime = Y \setminus \{y_0\}$. Пусть $f \colon X \times Y^\prime \to \MR (\MC)$. Это функция двух переменных $f(x,y)$, но мы будем смотреть на это как на параметрическое семейство функций ($y$ как параметр): $x \mapsto f(x,y) \Rightarrow$  изучаем функции на $X$, запараметризованные параметром $y \in Y^\prime$.

\textbf{Примеры}: $f(x,y) = 	\sin{(xy)}, \, f(x,y) = e^{x + y}, \, f(x,y) = x^y$

Нас будет интересовать, что происходит с семейством функций, когда $y \to y_0$, к какой функции это семейство будет приближаться. Раньше, в качестве параметра были числа из $\MN$ и получалась последовательность функций, cейчас же будет произвольное метрическое пространство. Ранее было так: 
$$
	f_n(x) = f(x,n), \, Y^\prime = \MN
$$ 
Можем ли мы придумать такое пространство $Y$, чтобы сходимость на нем была тем же самым, что и сходимость при  $n \to \infty$? Можно доопределить $Y$ до $Y = \MN \cup \{+\infty\}$, где $y_0 = +\infty$ и взять метрику:
$$
	\rho(n,m) = |\arctg{(n)} - \arctg{(m)}|, \, \arctg{(\infty)} = \dfrac{\pi}{2} \Rightarrow n \to \infty \Leftrightarrow \rho(n, \infty) \to 0 
$$
Ровно также, как это было для последовательностей, здесь будет два вида сходимости.

\begin{defn}
	Функция $f(x,y)$ \uwave{сходится поточечно} на $X$ к $\varphi(x)$ при $y \to y_0$, если:
	$$
		\forall x \in X, \, \lim\limits_{y \to y_0} f(x,y) = \varphi(x)
	$$
	или в терминах \uwave{предела по Коши}:
	$$
		\forall \VE > 0, \, \exists \, \delta > 0 \colon \forall y \in Y^\prime, \, 0 < \rho(y,y_0) < \delta \Rightarrow |f(x,y) - \varphi(x)| < \VE
	$$
	или в терминах \uwave{предела по Гейне}:
	$$
		\forall y_n \to y_0, \, y_n \neq y_0, \, f(x,y_n) \xrightarrow[n \to \infty]{}\varphi(x)
	$$
\end{defn}

\begin{defn}
	Функция $f(x,y)$ \uwave{сходится равномерно} на $X$ к $\varphi(x)$ при $y \to y_0$, если:
	$$
		\lim\limits_{y \to y_0}h(y) = \lim\limits_{y \to y_0}\sup\limits_{x \in X}|f(x,y) - \varphi(x)| = 0
	$$
	или в терминах \uwave{предела по Коши}:
	$$
		\forall \VE > 0, \, \exists \, \delta > 0 \colon \forall x \in X, \, \forall y \in Y^\prime, \, 0 < \rho(y,y_0) < \delta \Rightarrow |f(x,y) - \varphi(x)| < \VE
	$$
	Обозначается следующим образом: 
	$$
		f(x,y) \uconvm{X}{y \to y_0} \varphi(x)
	$$
\end{defn}
Видно, что это переход от предела последовательности к пределу функции, только теперь рассматривается функция от $y$ и у неё берется предел при $y \to y_0$.

\newpage
\begin{theorem}
	$f(x,y) \uconvm{X}{y \to y_0} \varphi(x) \Leftrightarrow \forall y_n \to y_0, \, y_n \neq y_0, \, f(x,y_n) \uconvm{X}{n \to \infty}\varphi(x)$.
\end{theorem}
\begin{proof}
	Это определение $\lim\limits_{y \to y_0}h(y) = 0$ по Гейне. То есть, предел по Коши $\Leftrightarrow$ предел по Гейне для $h(y)$.
\end{proof}

\begin{theorem}(\textbf{о перестановке пределов})
	Пусть $X$ - метрическое пространство и $x_0$ - его предельная точка, $y_0$ - предельная точка $Y$. Обозначим: $X^\prime = X \setminus \{x_0\}$, $Y^\prime = Y \setminus \{y_0\}$. Если выполнены условия:
	$$
		\forall y \in Y^\prime, \, \exists \, \lim\limits_{x \to x_0}f(x,y) = b(y) \wedge f(x,y) \uconvm{X^\prime}{y \to y_0} \varphi(x)
	$$
	то $\exists \, \lim\limits_{y \to y_0}b(y)$ и $\exists \, \lim\limits_{x \to x_0}\varphi(x)$ и эти пределы равны. Или по-другому:
	$$
		\lim\limits_{y \to y_0}\lim\limits_{x \to x_0}f(x,y) = \lim\limits_{x \to x_0}\lim\limits_{y \to y_0}f(x,y)
	$$
\end{theorem}
\begin{proof}
	Пусть $y_n \to y_0, y_n \neq y_0$, тогда $f(x,y_n) \uconvm{X}{n \to \infty} \varphi(x)$, при этом $\forall n, \, \exists \, \lim\limits_{x \to x_0}f(x,y_n) = b(y_n)$. По теореме для последовательностей (см. лекцию $12$): 
	$$
		\exists \, \lim\limits_{n \to \infty} b(y_n), \, \exists \, \lim\limits_{x \to x_0}\varphi(x) \wedge \lim\limits_{n \to \infty} b(y_n) = \lim\limits_{x \to x_0}\varphi(x)
	$$
	По определению предела функции по Гейне $\exists \, \lim\limits_{y \to y_0}b(y)$, который равен $\lim\limits_{x \to x_0}\varphi(x)$.
\end{proof}
\begin{corollary}
	Пусть $X, Y$ - метрические пространства, $x_0 \in X$, $y_0$ - предельная точка $Y$. Функция $f(x,y) \uconvm{X}{y\to y_0}\varphi(x)$. Если $\forall y \neq y_0$, $f(x,y)$ непрерывна в точке $x_0$, то $\varphi(x)$ непрерывна в точке $x_0$.
\end{corollary}
\begin{proof}
	Доказательство идентично тому, что было в лекции $12$.
\end{proof}
\begin{corollary}
	Возьмем пространство $X \times Y$ и введем метрику $\rho((x,y), (u,v)) = \rho_x(x,u) + \rho_y(y,v)$. В условиях теоремы, при $(x,y) \in X^\prime \times Y^\prime \subset X \times Y$, будет существовать двойной предел:
	$$
		\exists \, \lim\limits_{(x,y) \to (x_0, y_0)}f(x,y) = \lim\limits_{y \to y_0}\lim\limits_{x \to x_0}f(x,y) = \lim\limits_{x \to x_0}\lim\limits_{y \to y_0}f(x,y) = A
	$$
\end{corollary}
\begin{proof}
	Поскольку у $\varphi(x)$ есть равномерная сходимость, то:
	$$
		|f(x,y) - A| \leq |f(x,y) - \varphi(x)| + |\varphi(x) - A|
	$$
	Пусть $\VE > 0$, найдем такое $\delta_1 > 0$, что как только $0 < \rho_y(y,y_0) < \delta_1$, то $\forall x\in X^\prime, \, |f(x,y) - \varphi(x)| < \VE$. Выберем $\delta_2$ так, что когда $0 < \rho_x(x,x_0) < \delta_2$, то $|\varphi(x) - A| < \VE$, тогда:
	$$
		\forall \VE > 0, \, \exists \, \delta = \min\{\delta_1, \delta_2\} \colon 0 < \rho((x,y),(x_0,y_0)) < \delta \Rightarrow |f(x,y) - A| < 2\VE
	$$
\end{proof}
\begin{rem}
	Арифметика пределов переносится аналогично через определение предела по Гейне. Смотри лекцию $10$.
\end{rem}

\begin{theorem}
	Пусть $X = (\alpha,\beta)$, $Y$ - метрическое пространство, $y_0$ - предельная точка в $Y$, $Y^\prime = Y \setminus \{y_0\}$. Пусть задана функция $f(x,y) \colon X \times Y^\prime \to \MR$. Если $\forall y \in Y^\prime, \, x \mapsto f(x,y)$ - дифференцируема на $(\alpha,\beta)$, производная по аргументу $x$: 
	$$
		\dfrac{\partial f}{\partial x}(x,y) = f_x^\prime(x,y) \uconvm{X}{y \to y_0}g(x)
	$$ 
	и $\exists \, x_0 \in (\alpha, \beta) \colon \exists \, \lim\limits_{y \to y_0}f(x_0, y)$, то тогда:
	$$
		f(x,y) \uconvm{X}{y \to y_0}\varphi(x)
	$$
	где $\varphi(x)$ - дифференцируема на $X$ и $\varphi^\prime(x) = g(x)$.
\end{theorem}
\begin{proof}
	Пусть $y_n \to y_0, y_n \neq y_0$, тогда: 
	\begin{enumerate}[label=(\arabic*)]
		\item $\exists \, \lim\limits_{n \to \infty}f(x_0, y_n)$;
		\item $f_x^\prime(x,y_n) \uconvm{X}{n \to \infty}g(x)$;
	\end{enumerate}
	Следовательно, по теореме для последовательностей мы знаем: $f(x,y_n) \uconvm{X}{n \to \infty} \varphi(x)$, где $\varphi(x)$ будет дифференцируема и $\varphi^\prime(x) = g(x)$. Осталось показать, что сходимость не будет зависеть от выбора последовательности, сходящейся к $y_0$. 
	
	Знаем, что $\varphi(x_0) = \lim\limits_{y \to y_0}f(x_0,y)$, поскольку переход к последовательности вычисляет тот же самый предел $\Rightarrow$ это фиксированное значение. Более того, поскольку $\varphi^\prime(x) = g(x)$, то двух таких функций быть не может:
	$$
		(\varphi_1 - \varphi_2)^\prime = 0 \wedge \varphi_1(x_0) - \varphi_2(x_0) = 0 \Rightarrow \varphi_1 - \varphi_2 \equiv 0
	$$
\end{proof}
\begin{rem}
	Заметим, что ограниченность $X$ в данной теореме - важна, поскольку используется в теореме о дифференцируемости в последовательностях.
\end{rem}

\begin{theorem}
	Пусть $X = [\alpha, \beta]$, $Y$ - метрическое пространство, $y_0$ - предельная точка $Y$, $Y^\prime = Y \setminus \{y_0\}$. Если $\forall y \in Y^\prime$ функция $f(x,y)$ интегрируема по Риману на отрезке $[\alpha,\beta]$ относительно $x$ и верно:
	$$
		f(x,y) \uconvm{X}{y \to y_0} \varphi(x)
	$$
	Тогда функция $\varphi(x)$ интегрируема по Риману и выполняется следующее равенство:
	$$
		\ddint{\alpha}{\beta}\varphi(x) dx = \lim\limits_{y \to y_0}\ddint{\alpha}{\beta}f(x,y)dx
	$$
\end{theorem}
\begin{proof}
	Пусть $y_n \to y_0, y_n \neq y_0$, тогда $f(x,y_n) \uconvm{X}{n \to \infty}\varphi(x)$ и применяем теорему для последовательностей:
	$$
		\ddint{\alpha}{\beta}\varphi(x) dx = \lim\limits_{n \to \infty}\ddint{\alpha}{\beta}f(x,y_n)dx
	$$
	Если это выполнено для любой укзанной последовательности $\{y_n\}$, то значит для этого предела выполнено определение Гейне.
\end{proof}


\begin{theorem}(\textbf{Арцела}):
	Пусть $X = [\alpha, \beta]$, $Y$ - метрическое пространство, $y_0$ - предельная точка $Y$, $Y^\prime = Y \setminus \{y_0\}$. Пусть выполняются следующие условия:
	\begin{enumerate}[label=(\arabic*)]
		\item  $\forall y \in Y^\prime$ функция $f(x,y)$ интегрируема по Риману на отрезке $[\alpha,\beta]$ относительно $x$;
		\item функция $\varphi(x)$ интегрируема на $[\alpha, \beta]$;
		\item $\forall x \in X,\, f(x,y) \xrightarrow[y \to y_0]{} \varphi(x)$;
		\item $\exists \, C > 0 \colon |f(x,y)| \leq C, \, \forall x \in X, \, \forall y \in Y^\prime$;
	\end{enumerate}
	Тогда выполняется следующее равенство:
	$$
		\ddint{\alpha}{\beta}\varphi(x) dx = \lim\limits_{y \to y_0}\ddint{\alpha}{\beta}f(x,y)dx
	$$
\end{theorem}
\begin{proof}
	Пусть $y_n \to y_0, y_n \neq y_0$, по теореме Арцела для последовательностей мы получим:
	$$
			\ddint{\alpha}{\beta}\varphi(x) dx = \lim\limits_{n \to \infty}\ddint{\alpha}{\beta}f(x,y_n)dx
	$$
	Поскольку мы имеем здесь предел по Гейне, то получаем требуемое.
\end{proof}
\begin{theorem}(\textbf{Критерий Коши}): Функция $f(x,y)$ сходится равномерно на $X$ при $y \to y_0$ тогда и только тогда, когда выполняется условие Коши:
$$
	\forall \VE > 0, \, \exists \, \delta > 0 \colon \forall y_1, y_2 \in Y^\prime, \, 0 < \rho(y_1, y_0) < \delta \wedge 0 < \rho(y_2, y_0) < \delta \Rightarrow \sup\limits_{x \in X}\left|f(x,y_1) - f(x,y_2)\right| <\VE
$$
\end{theorem}
\begin{proof}\hfill\\
	($\Rightarrow$) Пусть предел $\lim\limits_{y \to y_0} \sup\limits_{x \in X}|f(x,y) - \varphi(x)| = 0$ существует. Тогда по определению:
	$$
		\forall \VE > 0, \, \exists \, \delta > 0 \colon \forall x \in X, \, \forall y \in Y^\prime, \, 0 < \rho(y,y_0) < \delta \Rightarrow |f(x,y) - \varphi(x)| < \VE
	$$
	Возьмем $\dfrac{\VE}{2} > 0$ и найдем $\delta > 0$ такое, что: $0 < \rho(y_1,y_0) < \delta \wedge 0 < \rho(y_2,y_0) < \delta$, тогда:
	$$
		\forall x \in X, \,  |f(x,y_1) - \varphi(x)| < \dfrac{\VE}{2} \wedge |f(x,y_2) - \varphi(x)| < \dfrac{\VE}{2} \Rightarrow \forall x \in X, \,  |f(x,y_1) - f(x,y_2)| =
	$$ 
	$$
		= |f(x,y_1) - \varphi(x) + \varphi(x) - f(x,y_2)| \leq |f(x,y_1) - \varphi(x)| + |f(x,y_2) - \varphi(x)| < \VE
	$$
	
	($\Leftarrow$) Пусть $y_n \to y_0, y_n \neq y_0$, тогда 
	$$
		\exists \, N \colon \forall n,m > N,\, 0 < \rho(y_n, y_0) < \delta \wedge 0 < \rho(y_m, y_0) < \delta
	$$ 
	значит применимо неравенство из условия Коши:
	$$
		\sup\limits_{x \in X}|f(x,y_n) - f(x,y_m)| < \VE
	$$
	Последовательность $f(x,y_n)$ удовлетворяет условию Коши (см. лекцию $10$) $\Rightarrow f(x,y_n)\uconvm{X}{n \to \infty}\varphi(x)$. Докажем единственность. Пусть $\exists \, z_n \to y_0, z_n \neq y_0$, перемешаем последовательности и рассмотрим новую:
	$$
		u_n \colon y_1, z_1, y_2, z_2, \dotsc , y_n, z_n, \dotsc \Rightarrow u_n \to y_0, u_n \neq y_0 \Rightarrow \exists \, f(x,z_n)\uconvm{X}{n \to \infty}\varphi(x)
	$$
	Поскольку предел существует, то существуют пределы подпоследовательности и они равны пределу последовательности.
\end{proof}

\subsection*{Пример равномерной сходимости семейства функций}
Рассмотрим пример, когда появляется естественным образом равномерная сходимость семейства функций. Возьмем функцию $f(x,y) \in C\left([a,b]\times [c,d]\right)$.

\begin{prop}
	$\forall y_0 \in [c,d], \, f(x,y) \uconvm{[a,b]}{y \to y_0} f(x,y_0)$.
\end{prop}
\begin{proof}
	Поскольку $f(x,y)$ - непрерывна на компакте, то по теореме Кантора $f(x,y)$ - равномерно непрерывна на нём, тогда:
	$$
		\forall \VE > 0, \, \exists \, \delta > 0 \colon \forall (x,y), (x_0, y_0),\, \sqrt{|x - x_0|^2 + |y - y_0|^2} < \delta \Rightarrow |f(x,y) - f(x_0,y_0)| < \VE
	$$
	Таким образом, если $x = x_0$, то достаточно, чтобы $|y - y_0| < \delta$, тогда:
	$$
		x = x_0 \Rightarrow \sqrt{|x - x_0|^2 + |y - y_0|^2} = |y - y_0| < \delta \Rightarrow 
	$$
	$$
		\Rightarrow \forall x \in [a,b], \, |f(x, y) - f(x, y_0)| < \VE \Rightarrow \sup\limits_{x \in [a,b]}|f(x,y) - f(x,y_0)| < \VE
	$$
\end{proof}
\begin{prop}
	Пусть $f(x,y)$ - непрерывна по $x$ и по $y$ в отдельности на $[a,b]\times [c,d]$ и $\forall y_0, \, f(x,y) \uconvm{[a,b]}{y \to y_0} f(x,y_0)$, тогда $f \in C([a,b] \times [c,d])$ - функция $f$ непрерывна по совокупности переменных на этом прямоугольнике.
\end{prop}
\begin{proof}
	Пусть $(x_2, y_2) \to (x_1, y_1)$ - фиксированная точка, тогда распишем следующую разность:
	$$
		|f(x_1, y_1) - f(x_2, y_2)| \leq |f(x_1,y_1) - f(x_2, y_1)| + |f(x_2, y_1) - f(x_2, y_2)|
	$$
	Зафиксируем $\VE > 0$, пусть $y_0 = y_1$, тогда $\exists \, \delta > 0 \colon |y_2 - y_1| < \delta \Rightarrow$ мы получим:
	$$
		\forall x \in [a,b], \, |f(x, y_1) - f(x, y_2)| < \VE\Rightarrow\forall x_2 \in [a,b], \, |f(x_2, y_1) - f(x_2, y_2)| < \VE
	$$ 
	Из непрерывности по $x$, как только $|x_2 - x_1| < \delta$, то при фиксированном $y = y_1$, мы сразу получим:
	$$
		|f(x_2,y_1) - f(x_1,y_1)| < \VE
	$$
	Следовательно:
	$$
		\forall \VE > 0, \, \exists \, \delta > 0 \colon \sqrt{|x_2 - x_1|^2 + |y_2 - y_1|^2} < \delta \Rightarrow |x_2 - x_1| < \delta \wedge |y_2 - y_1| < \delta \Rightarrow
	$$
	$$
		\Rightarrow |f(x_1, y_1) - f(x_2, y_2)| \leq |f(x_1,y_1) - f(x_2, y_1)| + \sup\limits_{x_2 \in [a,b]}|f(x_2, y_1) - f(x_2, y_2)| < \VE + \VE = 2 \VE
	$$
	Заметим, что сначала искали $\delta$ для переменной по которой есть равномерная сходимость.
\end{proof}

\end{document}